\section{Brute force}

\subsection*{Motivating problem}

\subsubsection*{Description}

\subsubsection*{Constraints}

\subsubsection*{Input}

\subsubsection*{Output}

\subsubsection*{Sample input}

\subsubsection*{Sample output}

\subsection*{Discussion}

A brute force problem is one in which the obvious algorithm is good enough; sometimes a more efficient solution is possible but more complex than the brute force approach. When reading a problem, it is often helpful to come up with a brute force approach even if the problem's contraints call for something faster. The brute force solution can serve as a starting point for the final solution. Of course, when the constraints are small enough that the brute force solution should run in time, no more thinking is necessary. Once you identify a problem as brute force, it becomes an implementation problem.

The primary hint that indicates that a problem should be solved with a brute force approach is the input size.

\subsection*{Practice problems}

\begin{itemize}
\item \url{http://codeforces.com/problemset/problem/629/A}
\item \url{http://codeforces.com/problemset/problem/631/A}
\item \url{http://codeforces.com/problemset/problem/626/A}
\item \url{http://codeforces.com/problemset/problem/464/B}
\item \url{http://codeforces.com/problemset/problem/124/B}
\end{itemize}

\subsection*{Further reading}
