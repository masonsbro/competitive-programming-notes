\section{Implementation}

\subsection*{Motivating problem}

\subsubsection*{Description}

Liam was hired by Moogle to help build the Gaps app. Moogle Gaps is an application for finding the length of the shortest path between two locations, using different modes of transportation. Liam has been assigned the part of the application for hot air balloons. Because hot air balloons are not constrained by roads, the shortest distance between two points via hot air balloon is the Euclidian distance between those two points (as X and Y coordinates).

Liam has spent his days watching TouYube videos, and the launch date for Gaps is fast approaching. He needs your help to implement the hot air balloon distance functionality. The app takes as input a list of $N$ locations and $Q$ queries. A location is given as a name (1-20 uppercase letters) and a pair of coordinates. A query asks for the shortest hot air balloon distance between two named locations.

\subsubsection*{Constraints}

\begin{gather*}
1 \leq T \leq 5 \\
1 \leq N \leq 1,000 \\
1 \leq Q \leq 5,000 \\
-10,000 \leq X_i \leq 10,000 \\
-10,000 \leq Y_i \leq 10,000
\end{gather*}

\subsubsection*{Input}

The first line contains $T$, the number of test cases. $T$ test cases follow.

The first line of each test case contains $N$ and $Q$. The next $N$ lines contain descriptions of the locations in the app. The $i$th line contains the name for the $i$th location as well as $X_i$ and $Y_i$ (which are integers). The next $Q$ lines contain queries. The $j$th query contains two names of locations.

\subsubsection*{Output}

For each test case, output $Q$ lines. The $j$th line contains the hot air balloon distance between the two locations in the $j$th query. An answer is considered correct if its absolute or relative error is less than $10^{-6}$.

\subsubsection*{Sample input}

\begin{verbatim}
1
4 5
VOUNTAINMIEW 45 100
AOSLNGELES -1000 350
YEWNORK 4000 5008
AELTVIV 9999 -4046
VOUNTAINMIEW AOSLNGELES
YEWNORK AELTVIV
VOUNTAINMIEW AELTVIV
AOSLNGELES VOUNTAINMIEW
AOSLNGELES YEWNORK
\end{verbatim}

\subsubsection*{Sample output}

\begin{verbatim}
1074.4882502847577
10861.07347364891
10782.923165821037
1074.4882502847577
6833.517688570068
\end{verbatim}

\subsection*{Discussion}

An implementation problem is one in which the correct approach is fairly obvious, and the main difficulty is in implementing it. In many competitions, the first problem or two are implementation or brute force (covered in the next tutorial) problems.

For this problem, it is clear that we should read in the $N$ locations and store their coordinates in some kind of associative data structure (say, a hash table or binary search tree). Then we can perform lookups in our data structure for each of the two locations in a query and calculate the Euclidean distance between them.

First, let's write a function to calculate the distance between two points. This is a direct translation of the formula $d = \sqrt{(x_1 - x_2)^2 + (y_1 - y_2)^2}$.

\begin{minted}{python}
def distance(a, b):
    return ((a[0] - b[0]) ** 2 + (a[1] - b[1]) ** 2) ** 0.5
\end{minted}

Now, let's read the input and store it in an associative data structure. I chose a hash table (Python's dictionary) because it is the fastest choice and it is very easy to use. In this problem, using a hash table leads to a solution that runs in $\mathcal{O}(N + Q)$ time, a balanced BST solution will run in $\mathcal{O}(N \log N + Q \log N)$, and a solution using a linked list, array, or unbalanced BST will run in $\mathcal{O}(NQ)$ time. Clearly, the hash table is the fastest, but for the small constraints of the problem all three will run in time.

\begin{minted}{python}
N, Q = map(int, raw_input().split())
locations = {}
for n in range(N):
    L, X, Y = raw_input().split()
    locations[L] = (int(X), int(Y))
\end{minted}

Now all we need to do is read the queries, process them, and print the output.

\begin{minted}{python}
for q in range(Q):
    L1, L2 = raw_input().split()
    print distance(locations[L1], locations[L2])
\end{minted}

And that's all there is to it! Implementation problems are tests of your basic programming ability. Sometimes they can be tricky and time-consuming, but they are usually not \textit{difficult}. The full code, along with some tests, can be found in the accompanying code directory.

\subsection*{Practice problems}

\begin{itemize}
\item \url{http://codeforces.com/problemset/problem/614/A}
\end{itemize}

\subsection*{Further reading}

There's not much to read about implementation problems. The best way to get better at them is to practice. I don't recommend taking much time to practice these outside of competitions, though---your time could better be spent practicing difficult problems. If you consistently run into trouble with implementation problems in competitions, start doing Div 2 A problems on Codeforces until you feel comfortable solving them in under 20 minutes.
